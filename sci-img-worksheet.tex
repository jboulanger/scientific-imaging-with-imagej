\documentclass[xcolor=table,DIV=19,twocolumn,10pt]{scrartcl}

\usepackage{lmodern}
\renewcommand\familydefault{\sfdefault}
\usepackage{beamerarticle}

\definecolor{EnglishViolet}{HTML}{4B3D60}
\definecolor{Crimson}{HTML}{DB162F}

\setbeamercolor{structure}{fg=EnglishViolet}
\setbeamercolor{normal text}{fg=white!25!black}

\usefonttheme{structurebold}
\setbeamercolor{footnote}{fg=gray}
\setbeamercolor{alerted text}{fg=Crimson}
\setbeamerfont{alerted text}{series=\bfseries}

\setbeamertemplate{footline}[frame number]
\setbeamertemplate{page number in head/foot}[appendixframenumber]
\setbeamertemplate{itemize items}[circle]

\usepackage{biblatex}
\addbibresource{references.bib}
\renewbibmacro{in:}{} % remove In
\renewbibmacro{title}{} % remove
\renewbibmacro{doi}{} % remove title

\newrobustcmd*{\footfullcitenomark}{%
  \AtNextCite{%
    \let\thefootnote\relax\let\mkbibfootnote\mkbibfootnotetext}%
  \footfullcite}

\usepackage{graphicx}
\graphicspath{{figs/}}

\usepackage{tikz,pgfplots}
\usetikzlibrary{calc,shapes,positioning}
\pgfplotsset{compat=1.15}
\pgfkeys{/pgfplots/tuftelike/.style={
    semithick,
    tick style={major tick length=4pt,semithick,black},
    separate axis lines,
    axis x line*=bottom,
    axis x line shift=10pt,
    xlabel shift=5pt,
    axis y line*=left,
    axis y line shift=10pt,
    ylabel shift=5pt}}

\usepackage{menukeys}

\theoremstyle{definition}
\newtheorem{exercice}{Exercise}

\title{Scientific Imaging with ImageJ}
\subtitle{Worksheet}
\date{\today}
\begin{document}
\maketitle

\begin{exercice} \textbf{Installation \& Update sites}
  To perform the exerces we will first install Fiji and the necessary packages.
  \begin{itemize}
  \item \url{https://imagej.net/software/fiji/downloads}
  \item Activate the following update sites: IJPB-plugins, CSBDeep, Stardist
  \item Restart Fiji 
  \end{itemize}

\end{exercice}

\begin{exercice} \textbf{Image is data}
  \begin{enumerate}
  \item Open a sample image : \menu{File>Open Sample>Blobs}
  \item Click on the \textcolor{red}{\texttt{>>}} icon in the FIJI toolbar and tick
        \menu{Pixel Inspector}
  \item Move your cursor over the image and observe the change of values in the FIJI toolbar. Now click on the image. What do you see?
  \end{enumerate}
\end{exercice}

\begin{exercice} \textbf{Brightness and contrast}
  \begin{enumerate}
  \item Open the boat example image \menu{File>Open Samples>Boats}
  \item Press the short cut \keys{\Alt+\shift+C} or go to \menu{Image>Adjust>Brightness/Contrast}
  \item Change the minimum and maximum sliders and observe the image
  \item Press \keys{Set} and set the Minimum displayed value to 50
    and the maximum displayed value to 200, and press the \keys{OK} button.
  \item Finally press \keys{Apply} and observe the change of the
    histogram in the B\&C window.
  \end{enumerate}
\end{exercice}

\begin{exercice} \textbf{Multi-channel images}
  \begin{enumerate}
  \item Open a composite image \menu{File>Open Samples>Fluorescent
        Cells}
  \item Use the channel tool (\menu{Image>Color>Channel Tool...}) to display each colour individually. Try different settings and observe the changes in the image.
  \item Use \menu{Image>Color>Split Channels} to split the stack into separate images
  \item Use \menu{Image>Color>Merge Channels} to create a new image consisting of the red and green channel only.
  \end{enumerate}
\end{exercice}

\begin{exercice} \textbf{Scale bar}
  \begin{enumerate}
  \item Open the sample image \menu{File>Open Samples>HeLa cells (48-bit RGB)}
  \item Inspect the metadata \keys{Command+I} (or \menu{Image>Show Info...}) and the pixel size \keys{\ctrl+\shift+P} (or \menu{Image>Properties...})
  \item Adjust the brightness and contrast for each channels
  \item Add a scale bar to the image
  \item Save the image and include the image in a presentation (powerpoint, keynote \dots)
  \end{enumerate}
\end{exercice}


\begin{exercice} \textbf{Look up table (LUTs)}
  \begin{enumerate}
  \item Open a grey scale image \menu{File>Open Samples>Boats}
  \item Apply a LUT to the image \menu{Image>Look Up Tables> 5
    Ramps}
  \item Duplicate the image \menu{Image>Duplicate...}
  \item Convert the new image to RGB \menu{Image>Type>RGB Color}
  \item Observe the values for each pixel and compare the Byte size of the two images. What is the difference?
  \item Open the blobs example \menu{File>Open Samples>Blobs} and
    comment on the LUT
  \item Explore other commands \menu{Image>Color>Display LUTs},
    \menu{Image>Color>Edit LUT}. (Note: \menu{Edit LUT} will only work if the active image  is not a RGB image. Make sure you click on the correct image before calling \menu{Image>Color>Edit LUT})
  \item Open the Fluorescent Cells sample and change the red channel
    to magenta.
  \end{enumerate}
\end{exercice}

\begin{exercice} \textbf{Thresholding \& Selection}
  \begin{enumerate}
  \item Open the Blobs example \menu{File>Open Samples>Blobs}.
  \item Open the tresholding tool \keys{\ctrl+\shift+T} or \menu{Image>Adjust>Threshold...}.
  \item Select a threshold, or a thresholding method.
  \item Convert the threhsolded region into a ROI using \menu{Edit>Selection>Create selection}.
  \item Open the ROI manager(\menu{Analyze>Tools>ROI Manager}) and add this selection to the ROI manager using \keys{t}.
  \end{enumerate}
  Here the resulting selection can be composed of several non-connected (composite) regions.
\end{exercice}

\begin{exercice} \textbf{Thresholding \& Particle analysis}
  \begin{enumerate}
  \item Open a grey scale image.
  \item Duplicate the image.
  \item Set a threshold on the duplicated image and make the image binary \menu{Process>Binary>Make Binary}. Another way of getting a binary mask is using \menu{Image>Adjust>Threshold...} and clicking \keys{Apply}.
  \item Add connected components to the ROI Manager using \menu{Analyze>Analyze Particles} (make sure you click \keys{Add to Manager} in the \menu{Analyze Particles} dialog).
  \item Measure the mean intensity of each ROI by clicking on \menu{Measure} in the ROI Manager. Navigate to \menu{Analyze>Set Measurements...} to specify the required measurements. In the \menu{Set Measurements} menu select the name of the image on which you would like to do the intensity measurements under \menu{Redirect to:}.
  \end{enumerate}
\end{exercice}

\begin{exercice} \textbf{Filtering, noise and edges}
  \begin{enumerate}
     \item Create an image with a light grey square on a darker grey
      square.
      \begin{enumerate}
          \item Create an empty image using \menu{File>New>Image...}
          \item \menu{Edit>Options>Colors...} Set the foreground to \menu{lightgray} and the background to \menu{darkgray}
          \item Draw a rectangle on the image using the rectangle tool on the Fiji toolbar.
          \item \menu{Edit>Clear Outside}
          \item \menu{Edit>Fill}
      \end{enumerate}
    \item Launch the histogram and press \menu{Live}. How does the histogram look like?
    \item Add 50 intensity units to every pixel in the image.  \menu{Process>Math>Add...} Observe what happens to the histogram.
    \item Add noise to the image. Choose a noise tool under   \menu{Process>Noise}. How does the histogram look like?
    \item Apply some filtering to the image (under \menu{Process>Filter} or \menu{Process>Smooth}) and observe the change in the histogram.
    \item Set a threshold and observe the result.
    \item  Experiment with different filters.
  \end{enumerate}
\end{exercice}

\begin{exercice} \textbf{Computing manually the Pearson correlation coefficient}
  \begin{enumerate}
    \item Open the image coloc.tif
    \item Split the channels, change the bit depth to 32 bit
    \item Measure the mean intensity and subtract it to both image
    \item Compute the product of the two images
    \item Measure the mean of the product and the standard deviation of each channel
    \item Compute the ratio of the mean of the product by product of the standard deviations.
    \item Compare with the result given by coloc2.
  \end{enumerate}
\end{exercice}

\begin{exercice} \textbf{Seeded watershed}
  \begin{enumerate}
    \item Open the "HeLa Cells (48-bit RGB)" example
    \item Duplicate the third channel and create a mask of the nuclei
    \item Compute the connected components (\menu{Plugins>MorpholibJ>Binary Images>Connected Components Labeling})
    \item Filter out smaller regions (\menu{Plugins>MorpholibJ>Label Images>Label Size Filtering}) and rename the image as "marker".
    \item Duplicate "marker" and threshold with values zeros and one and apply.
    \item Compute the distance transform \menu{Process>Binary>Distance Map} and rename the image as "input".
    \item Compute the watershed using \menu{Plugins>MorpholibJ>Segmentation>Marker-controlled Watershed} with "calculate dams" ticked.
  \end{enumerate}
\end{exercice}

\begin{exercice} \textbf{Stardist}
  \begin{enumerate}
    \item Open the "HeLa Cells (48-bit RGB)" example
    \item Duplicate the third channel 
    \item Run Stardist using \menu{Plugin>Stardist 2D}
    \item Select the "Versatile" model, "ROI Manager" for the output type. 
  \end{enumerate}
\end{exercice} 

\end{document}
