\documentclass[xcolor=table]{scrartcl}

\usepackage{beamerarticle}

\definecolor{EnglishViolet}{HTML}{4B3D60}
\definecolor{Crimson}{HTML}{DB162F}

\setbeamercolor{structure}{fg=EnglishViolet}
\setbeamercolor{normal text}{fg=white!25!black}

\usefonttheme{structurebold}
\setbeamercolor{footnote}{fg=gray}
\setbeamercolor{alerted text}{fg=Crimson}
\setbeamerfont{alerted text}{series=\bfseries}

\setbeamertemplate{footline}[frame number]
\setbeamertemplate{page number in head/foot}[appendixframenumber]
\setbeamertemplate{itemize items}[circle]

\usepackage{biblatex}
\addbibresource{references.bib}
\renewbibmacro{in:}{} % remove In
\renewbibmacro{title}{} % remove
\renewbibmacro{doi}{} % remove title

\newrobustcmd*{\footfullcitenomark}{%
  \AtNextCite{%
    \let\thefootnote\relax\let\mkbibfootnote\mkbibfootnotetext}%
  \footfullcite}

\usepackage{graphicx}
\graphicspath{{figs/}}

\usepackage{tikz,pgfplots}
\usetikzlibrary{calc,shapes,positioning}
\pgfplotsset{compat=1.15}
\pgfkeys{/pgfplots/tuftelike/.style={
    semithick,
    tick style={major tick length=4pt,semithick,black},
    separate axis lines,
    axis x line*=bottom,
    axis x line shift=10pt,
    xlabel shift=5pt,
    axis y line*=left,
    axis y line shift=10pt,
    ylabel shift=5pt}}

\usepackage{menukeys}

\theoremstyle{definition}
\newtheorem{exercice}{Excercice}

\title{Scientific Imaging with ImageJ}
\subtitle{Worksheet}

\begin{document}
\maketitle

\begin{exercice} Image is data
  \begin{enumerate}
  \item Open a sample image : \menu{File>Open Sample>Blobs}
  \item Click on the \textcolor{red}{\texttt{>>}} icon and tick
        \menu{Pixel Inspector}
  \end{enumerate}
\end{exercice}

\begin{exercice} Brightness and contrast
  \begin{enumerate}
  \item Open the boat example image \menu{File>Open Samples>Boats}
  \item Press the short cut \keys{\Alt+\shift+C}
  \item Change the minimum and maximum sliders and observe the image
  \item Press \keys{Set} and set the Minimum displayed value to 50
    and the maxium displayed value to 200, and press the \keys{OK} button.
  \item Finally press \keys{Apply} and obsere the change of the
    histogram in the B\&C window.
  \end{enumerate}
\end{exercice}

\begin{exercice} Scale bar
  \begin{enumerate}
  \item Open an image % TODO specify an image
  \item Inspect the metadata \keys{\Alt+I} and the pixel size \keys{\ctrl+\shift+P}
  \item Adjust the brighness and contrast for each channels
  \item Add a scale bar to the image
  \item Include the image in a presentation (powerpoint, keynote \dots)
  \end{enumerate}
\end{exercice}


\begin{exercice} Multi-channel images
  \begin{enumerate}
  \item Open a composite image \menu{File>Open Samples>Fluorescent
        Cells}
  \item Use the channel tool to display each colour individually
  \end{enumerate}
\end{exercice}

\begin{exercice} Look up table (LUTs)
  \begin{enumerate}
  \item Open a grey scale image \menu{File>Open Samples>Boats}
  \item Apply a LUT to the image \menu{Image>Look Up Tables> 5
    Ramps}
  \item Convert the image to RGB \menu{Image>Type>RGB Color}
  \item Observe the values at each pixel
  \item Open the blobs example \menu{File>Open Samples>Blobs} and
    comment on the LUT
  \item Explore other commands \menu{Image>Color>Display LUTs},
    \menu{Image>Color>Edit LUT}
  \item Open the Fluorescent Cells sample and change the red channel
    to magenta.
  \end{enumerate}
\end{exercice}

\begin{exercice} Thresholding \& Selection
  \begin{enumerate}
  \item Open the Blobs example \menu{File>Open Samples>Blobs}
  \item Open the tresholding tool \keys{\ctrl+\shift+T}
  \item Select a threshold, or a thresholding method
  \item Convert the mask into a ROI using \menu{Edit>Selection>Create selection}
  \item Add this selection to the ROI manager using \keys{t}
  \end{enumerate}
\end{exercice}

\begin{exercice} Thresholding \& ROIs
  \begin{enumerate}
  \item Open a grey scale image
  \item Set a threshold and make the image binary pressing
  \item Add connected components to the ROI Manager using \menu{Analyze>Analyze Particle}
  \item Measure the mean intensity of each ROI by clicking on \menu{Measure} in the ROI Manager.
  \end{enumerate}
\end{exercice}

\begin{exercice} Thresholding \& ROIs
  \begin{enumerate}
     \item Create an image with a light grey square on a darker grey
      square.
    \item Add noise to the image
    \item Launch the histogram and press \menu{Live}
    \item Apply some filtering to the image
    \item Set a threshold and observe the result
    \item  Experiment with different filters
  \end{enumerate}
\end{exercice}



\end{document}
